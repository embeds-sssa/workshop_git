%!TEX program = xelatex
\documentclass[10pt,svgnames]{beamer}

\PassOptionsToClass{svgnames}{xcolor}

\usetheme[progressbar=frametitle]{metropolis}
\usepackage{fontspec}
\setmainfont{Linux Biolinum O}
\setsansfont{Linux Biolinum O}
\setmonofont{Inconsolata}[Scale=MatchLowercase]
\usepackage{listings}

\lstset{%
basicstyle=\ttfamily,
  frame=single,
  xleftmargin=\parindent,
}

\graphicspath{{./figures/}}

\usepackage{appendixnumberbeamer}

\usepackage{booktabs}

% \usepackage{pgfplots}
% \usepgfplotslibrary{dateplot}

% \usepackage{xspace}
% \newcommand{\themename}{\textbf{\textsc{metropolis}}\xspace}

% Git puns: cogitate, digital, fugitive, legitimate, longitude, prestidigitation, regurgitate

\title{Git better}
\subtitle{Collaborative project management using \emph{Git} and \emph{Github}}
\date{\today}

\author{Matteo Sostero}
\institute{Sant'Anna School of Advanced Studies}
% \titlegraphic{\hfill\includegraphics[height=1.5cm]{logo.pdf}}

\begin{document}

\maketitle

% \begin{frame}{Table of contents}
%   \setbeamertemplate{section in toc}[sections numbered]
%   \tableofcontents[hideallsubsections]
% \end{frame}


\section{Version control with Git}

\begin{frame}
\frametitle{Git}
\includegraphics[height=1cm]{Git-logo}

Git is a distributed version control system for tracking changes in files and coordinating work on those files among multiple people.

\textbf{Version control} (Git, Mercurial, CVS, Subversion, Bitkeeper):
\begin{itemize}
   \item keep track of changes to a set of (text, binary) files over time locally;
   \item easily track what changed between any two versions (text);
   \item revert any change if needed;
   \item back up and distribute copies of files;
   \item collaborate on projects.
 \end{itemize} 

\textbf{Distributed}:
\begin{itemize}
  \item developers keep local copy of entire code and hsitory;
  \item can make changes offline and asynchronously;
  \item changes (easily) reconciled later.
\end{itemize}
\end{frame}


\begin{frame}
\frametitle{Git pros}

Advantages of Git:
\begin{itemize}
  \item widely used, supported, documented;
  \item online platforms (\href{https://github.com/}{Github}, \href{https://bitbucket.org/}{bitbucket}, \href{https://about.gitlab.com/}{Gitlab});
  \item dektop interfaces: shell, \href{https://desktop.github.com/}{Github Desktop}, \href{https://www.sourcetreeapp.com/}{SourceTree}, \href{https://www.gitkraken.com/}{GitKraken};
  \item integration in editors and IDEs: Emacs, Sublime Text, RStudio, XCode, Visual Studio, \ldots;
  \item distributed development;
  \item easy branching: eg, experimental branches for trying changes);
  \item makes complex merges quick.
\end{itemize}
\end{frame}


\begin{frame}
\frametitle{Git cons}
\label{git_cons}

Disadvantages:
\begin{itemize}
  \item steep learning curve;
  \begin{itemize}
      \item \hyperlink{xkcd_git}{complex conceptual model};
      \item cryptic man pages (but good documentation!).
  \end{itemize}  
  \item trivial handling of binary files, images, Office documents;
\end{itemize}

Required workflow changes:
\begin{itemize}
  \item keep project under single directory (\emph{outside} Dropbox, Google Drive, piCloud, \ldots!)
  \item consistency in personal and team coding style (eg, indentation, spacing, line-breaking);
  \item save and commit your changes manually and frequently;
  \item requires to document code and explain changes.
\end{itemize}
\end{frame}


\section{How does it work?}

\begin{frame}
\frametitle{Git concepts}
    
\begin{itemize}[<+->]
\item \textbf{repository} (\textbf{repo}): a collection of files and their history
\begin{itemize}
   \item where the code is kept: under a single root directory;
   \item can be \textbf{local} (your machine), or distributed across your team or on a \textbf{remote} server (eg, Github).
 \end{itemize}

\item \textbf{commit}: a snapshot of your files at a given time:
\begin{itemize}
  \item how Git keeps track of changes in code over time and across team;
  \item manually \textbf{add} one or more files to include to commit them and describe what changed in a message;
  \item a permanent record of what changed, when, with respect to what, by whom. Uniquely identified by a SHA-1 code (eg, \texttt{a12b34}).
\end{itemize}

\item \textbf{clone}: copy a remote repo on your machine;
\item \textbf{push}: send your local changes to the remote server;
\item \textbf{pull}: (\textbf{fetch}+\textbf{merge}): get the latest changes from the server.
\end{itemize}

\end{frame}



%% Basics: 
% spaces (stage, index, remote)
% interfaces
% history
% undoing everything
% branching

%% Workflow:
% what (not) to include in project
% sync with remote: push and pull
% tag model versions, release
% branching for features

%% Style:
% comment
% consistent style
% commit early and often, by milestones
% semantic commit messages
% never commit broken code

%% Collaboration
% forking, branching, pull request model
% issue tracking and comment
% project readme and collaboration guide

\begin{frame}
\frametitle{Github}
\end{frame}

\begin{frame}
\frametitle{Git model}
\input{git_workflow.fig}
Adapted from \href{https://tex.stackexchange.com/a/70332/14260}{https://tex.stackexchange.com/a/70332/14260}
\end{frame}

\begin{frame}[standout]
Thank you!
\end{frame}

\appendix

\begin{frame}
\label{xkcd_git}
\frametitle{XKCD on git \hfill\hyperlink{git_cons}{\beamergotobutton{back}}}

\begin{center}
\includegraphics[height=0.5\paperheight]{figures/xkcd-git.png}
\end{center}

\emph{If that doesn't fix it, git.txt contains the phone number of a friend of mine who understands git.
Just wait through a few minutes of “It's really pretty simple, just think of branches as\ldots” and eventually you'll learn the commands that will fix everything.
}

\end{frame}

\end{document}
